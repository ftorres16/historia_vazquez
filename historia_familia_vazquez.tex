\documentclass[a4paper, 12pt]{report}
\usepackage[utf8]{inputenc}
\usepackage[spanish]{babel}
\usepackage{booktabs}
\usepackage{verse}

\newcommand{\attrib}[1]{%
    \nopagebreak{\raggedleft\footnotesize #1\par}
}
\renewcommand{\poemtitlefont}{\normalfont\large\itshape\centering}

\title{Historia de la Familia Vázquez Eliseche}
\author{Escrito por Lucy Vázquez}
\date{}

\begin{document}

\maketitle
\tableofcontents

\chapter{Bisabuelos}

\section{Ciria Silva Oliveira}

Nació en Brasil (Yaguarón) aproximadamente en 1816.

Hija de Juan Ignacio Silva y de Claudia Oliveira. Falleció aproximadamente en 1881.

\section{Andrés Eliseche Idiarte}

Nació en Francia (Saint-Palais) aproximadamente en 1816.

Hijo de Pedro Eliseche y de Juana Idiarte. Falleció en Mayo de 1877.

\section{Tapera}

Hoy es solo eso: piedra, pasto, cielo. Ayer fue hogar.

Construida por el bisabuelo Andrés Eliseche para su hija Isabel.

El campo pertenecía a Antonio Vázquez, el abuelo. Se encuentra a algunos kms de María Albina.

Vendido por la abuela a Don Ramón Olascuaga, hoy pertenece a Sanz.

Se levanta sobre una loma. Mide aproximadamente $14\times6$~m. No quedan vestigios de divisiones interiores.

Piedra asentada en barro. La puerta de entrada se orienta hacia el suroeste; algunas cuadras al sur, los montes de Olimar.

Derruido escenario de amor y dolor.

Allí se tornan reales las anécdotas narradas por Chacha (Isabel Vázquez).

\begin{quote}
    Descendía una loma, llevando en mi delantal almidonado huevos de pájaros. Quería vaciarlos y enhebrar las cáscaras de colores para hacer un collar. Mi amiga Rosalía Velino, por hacerme una broma, gritó ¡una víbora!. Asustada apreté el delantal, y rompí en él todos los huevos.
\end{quote}

Revivía la escena ingenua, y mezclaba lágrimas de risa y de nostalgia.

\begin{quote}
    Papá no estaba. Éramos todos muy chicos, y estábamos solos con Mamá.

    La campaña era peligrosa; con frecuencia andaban malhechores, y más aún allí, tan cerca del monte. Era de noche. Se oyeron pasos afuera. Mamá apagó la vela y nos hizo callar. Comenzó a revisar puertas y ventanas. Al pasar la mano por una puerta, tocó la hoja de un cuchillo con el que trataban de abrir. ---¡Andrés, tráeme la escopeta!--- Andrés, el hijo mayor, era muy pequeño. El malhechor huye. No había escopeta.
\end{quote}

\begin{quote}
    Una de las hermanitas fallecidas era tan rubia, que los días de sol no podía salir afuera.
\end{quote}

\begin{quote}
    A veces pasaban ``turcos'' vendiendo telas, puntillas, broches. Yo tuve un vestido precioso que costó dos vintenes el metro de tela.
\end{quote}

\begin{quote}
    Se trabajaba mucho: una vez por semana se amasaba el pan; ha\-cí\-amos el jabón y las velas. El surtido se traía una vez por mes de Treinta y Tres.
\end{quote}

\begin{quote}
    Una vez pasó un hombre con un oso; lo hacía bailar. Después supimos que el oso abrazó al hombre y lo mató.
\end{quote}

\begin{quote}
    En el monte cazábamos mariposas. Eran de todos los tamaños y colores.

    Íbamos a pasar el día: Mamá preparaba comida para llevar. Ella lavaba, nosotros jugábamos.
\end{quote}

Al oír las narraciones dichas con emoción, nos transportábamos al pasado. Veíamos el monte poblado de pájaros que ponían huevos de extraordinario colorido, en nidos al alcance de la mano. Corríamos detrás de las mariposas con combinaciones cromáticas imposibles. Veíamos niñas con ropajes distintos, niños serios y obedientes.

Hoy la casa se ha hecho real. Sus paredes son palpables; de ellas extraje una piedra, que conservo: allí la puso el bisabuelo Andrés. Las anécdotas, los personajes, se tornan más interesantes.

Sucedió allí. Vimos la tapera recortada en el cielo, el mismo de entonces.

Tal vez ``Ellos'' nos acompañaron durante la visita.

Es posible que\ldots

\begin{verse}
    \ldots cuando brilla luna llena\\
    se ve a la madre joven\\
    que sonríe y canturrea\\
    Mirando está a sus hijitas\\
    que corren sobre la hierba.\\
    Juegan a que aún viven.\\
    Olvidan que yacen muertas\ldots
\end{verse}

\chapter{Antonio Vázquez Villanueva}

Un retrato en la sala. Serio, castaña barba cuidada.

Desde allí vio consumirse el tiempo de sus hijos, crecer a sus nietos. 

Silueta del pasado, delineada a través de narraciones y documentos.

Nació en Montevideo, aproximadamente en 1842. Fueron sus padres Miguel Vázquez\footnote{Civil asimilado al ejército, se ocupaba de hacer rondas militares.} y María Villanueva; sus abuelos paternos, Pedro Vázquez (español) y Dorotea Gutiérrez (española), sus abuelos maternos: Rosendo Villanueva y María Apolinaria Iglesias (español y uruguaya).

Se inició en la carrera de las armas, alcanzando el grado de capitán (inválidos) del coronel Latorre. Por su problema con éste\footnote{Problema con Latorre: un compañero de armas, José Romero, hizo mal una fiscalización, y temeroso de las consecuencias, culpó al abuelo. Cuando Latorre supo de la injusticia cometida, y había dictado sentencia, y no era persona capaz de enmiendas.}, fue degradado y enviado al departamento de Minas, como medida disciplinaria, debiendo el haber salvado la vida, a su parentesco con figuras de relevancia en el ambiente político.

\bigbreak{}

En 1877 hizo los trámites ante sus superiores para obtener la venia para contraer matrimonio con Isabel Eliseche Silva. Por esa época tenía ya varios hijos naturales, descendiendo de una de ellos la familia Lámpez, de Treinta y Tres.

Fue muy severo, pero también extremadamente cariñoso. Vivió en un paraje cercano a María Albina. Próximo a la terminación del siglo, en la última década, se radicó con su familia en la Villa de Treinta y Tres, adquiriendo la manzana rodeada por las calles Celedonio Rojas, Santiago Gadea, Andrés Areguatí y Basilio Araújo.

Allí, en el N$^o$~495 de C. Rojas, falleció el 12 de julio de 1904.

Activa vida política; militó en el Partido Colorado, siendo combatiente en las filas de su jefe y amigo, el coronel Basilicio Saravia.

Al radicarse en la villa, instaló una carnicería que ocupaba la esquina con Basilio Araújo; en ella fue mandadero Pedro Leandro Ipuche.

\section{Notas sobre Antonio Vázquez}

Eran épocas en las que el partidismo llevaba a las armas, pero no quebraba la amistad de los hombres de bien:

La familia Obaldía, militante en el partido blanco, era vecina y amiga de los Vázquez.

En una oportunidad el ejército blanco iba a pasar por el pueblo, y se temían disturbios. Los Obaldía pidieron protección a Antonio, y en su casa permanecieron hasta que pasó el peligro.


Un colorado protegió a una familia blanca\ldots del ejército blanco.

\bigbreak{}

Se preparaba un enfrentamiento bélico entre blancos y colorados. Antonio se encontraba en un comercio, cuando un parroquiano que allí estaba, comentó sobre ``qué jabón se iban a llevar los colorados''.

Nada dijo Antonio. Compró allí un huevo de avestruz, lo cocinó un poco, oradó la cáscara, y le entreveró pedazos de jabón. Regresando donde estaba el conversador\ldots le obligó a comerlo hasta lo último. Luego lo colocó sobre el caballo, y dejó que el instinto lo regresara a su casa\ldots sin nada de ganas de hablar!

\bigbreak{}

Cuando Antonio compró la casa en Treinta y Tres, no existían las habitaciones que hoy conforman la fachada; cuando las hizo construir, en la esquina, en la parte alta, lucía la fecha: 1896.

La cocina era una construcción de barro, techo de paja que cerraba el cuadro del patio.

\bigbreak{}

A algunos metros de la casa de Antonio, en Treinta y Tres vivía una mujer con su hija, a la que castigaba con crueldad. Con frecuencia los gritos de la niña alteraban la tranquilidad de la familia Vázquez.

Un día iba Antonio a caballo, y pasó frente a la vivienda; pudo ver a la niña atada, y a la madre pegándole encarnizadamente. Dejó el caballo, y sin que mediaran palabras ni reproches desató a la niña, ató en su lugar a la madre y le dio unos buenos fustazos.

\bigbreak{}

Su muerte: había tenido tifus, enfermedad temible en la época, y estaba muy delicado. El médico le dijo que ya el peligro había pasado, y él creyó que ya podía hacer su voluntad en todo. Pidió un puchero de gallina. Inútilmente intentaron disuadirle. El respeto a la autoridad del jefe de familia se impuso. Le sobrevino una peritonitis fatal.

\chapter{Isabel Eliseche}

Fueron sus padres Andrés Eliseche (francés) y Ciria Silva Oliveira (brasileña).

Nació en Olimar Chico el 29 de ¿noviembre? ¿diciembre? (contradicción en documentos) de 1857. Allí transcurrió su niñez y adolescencia, y allí unió su vida a la de Antonio Vázquez Villanueva, el 11 de julio de 1877.

Su padre, carpintero, construyó la casa de piedra e hizo los muebles.

Llegaron los hijos: Jesús, Andrés, Siria, Prima, Ema, Isabel.

Cuando había nacido Andrés\footnote{El original decía ``los dos mayores'', pero después lo tacha por ``Andrés''}, se abatió una epidemia de viruela sobre el lugar. La madre, Ciria Silva, contrae la enfermedad. (Andrés Eliseche había fallecido en mayo de 1877, dos meses antes de la boda de Isabel). Isabel se traslada para cuidarla, y con dolor la sepulta en el monte. Se contagia, pero sobrevive.\footnote{El original acá dice ``no'' seguido de una cruz ($\times$)}

Años más tarde, otra enfermedad: la difteria, marcará no el cuerpo, sino el corazón de Isabel y Antonio: en sólo 9 días fallecieron Ciria, Prima (Antonia) y Ema. Los montes de Olimar recogieron el dolor de la madre, que se refugiaba en ellos para gritar, llamando impotente a sus hijitas muertas.

Antonio resuelve trasladar la familia, enlutada y herida, a Treinta y Tres. Los tres hijos sobrevivientes van enfermos. Ensayan en ellos la vacuna, recién llevada para ser probada en las tropas. La muerte se aleja.

Nacen otros hijos en los que repiten nombres: Ernesto, Siria, Ema, Luis Antonio.

Y llega la revolución. Temor. Antonio y Jesús son combatientes.

El hogar se transforma: preparan vendas, ayudan a cuidar los heridos que llegan en carretas. A lo lejos, se oyen los cañonazos.

Otra vez la zarpa cruel de la enfermedad y de la muerte: tuberculosis; Andrés. ¡Sólo 25 años! Es un nuevo dolor, inesperado y desgarrante. Y luego otro: Antonio, el compañero sufrido y fuerte.

Cuando enferma, Jesús se encuentra perdido en el frente de batalla: se lo cree muerto. Pero regresa con tiempo apenas de asistir al fin de su padre.

Isabel se sobrepone: tiene aún hijos pequeños que requieren su atención.

Pasa el tiempo; la familia se expande. Jesús, casado, tiene 5 hijos. Pero también ahí hay dolor: Julio, el penúltimo, es sordomudo.

El corazón de la madre ya pide descanso, pero aún debe sufrir más pruebas. Ema, la menor de las mujeres, ha sido operada de cáncer. También lo sufre Jesús, que fallece en 1935.

En Isabel se han agotado las fuerzas. El 30 de enero de 1936, encuentra por fin la paz.

Cuentan quienes la conocieron, que pese a los repetidos y duros golpes, conservó siempre dulzura y buen humor. Amiga fiel, sociable, gustaba de reuniones. Los domingos, después de asistir a Misa, con frecuencia visitaba o recibía.

En una oportunidad fue con sus hijos, en carruaje, a pasar unos días a casa de una familia amiga que vivía en Olimar Chico. No había puente sobre el río. Llovió mucho, éste creció, ¡y debieron quedar cerca de un mes! Veía cómo se terminaban las provisiones; el gritarlo la divertía.

Gran corazón. Estando muy pobre una nieta de Antonio (hija de una hija natural), la llevó para su casa, junto con su numerosa prole, ocupándose con energía de conseguir trabajo para el marido.

\section{Notas sobre Isabel Eliseche}

Isabel Vázquez Brovia leyó lo que escribí sobre la abuela Isabel Eliseche. No comparte lo que anoté sobre su ``buen humor''. Sus recuerdos están vivos; yo no la conocí. Me dice que era una mujer profundamente triste: nunca reía. Sí tenía un carácter dulce: nunca una queja ni una palabra fuerte.

Me contó algunas anécdotas:

\bigbreak{}

Era ya grande, y no sabía leer ni escribir: vivía en campaña, y también sus padres eran analfabetos.

En una oportunidad, una persona que les visitaba notó su aguda inteligencia, y la inició en el aprendizaje. A los 3 meses leía y escribía.

Fue eso toda la enseñanza que recibió, pero fue larga la práctica: gustaba de leer extensas novelas que se publicaban por capítulos, y que pacientemente coleccionaba.

Tal vez recordando sus propias carencias, otorgó siempre gran importancia a la escolaridad. Empleaba muchachos para que hicieran los mandados y ayudaran en las tareas de la quinta, y exigía con firmeza a los padres que les enviasen a la escuela, en una época en que se concedía escasa importancia a la obligatoriedad de la educación.

Prohibido en su casa: tocar los nidos de golondrinas y ratoneras, partir el pan con la mano, hablar mal de las personas.

Eran escasos sus haberes y grande su generosidad. Mensualmente cobraba una modesta pensión por el abuelo Antonio, con la que ayudaba a su nuera Rosa y a sus cinco nietos. Uno de los proveedores de la familia, un Sr. Rivero, que en su juventud fuera hacendado, y luego tuvo que vender verduras y leche para subsistir, al llegar a la casa pasaba a la cocina, donde le esperaba a diario un abundante desayuno.

Fernando Lámpez --- nieto de abuelo Antonio--- cuando tenía problemas e\-co\-nó\-mi\-cos encontraba en el comedor de la abuela Isabel lo necesario para alimentarse hasta que llegaban épocas mejores.

Semestralmente cobraba la renta del campo. Hacía entonces un surtido grande para la familia, y 2 o 3 más modestos para algunos protegidos.

Sus nietos adoraban a ``Mamá Isabel''. Los niños se turnaban para dormir con ella, a los pies de la cama. Los esperaba con galletitas y chocolate, y al día siguiente los enviaba a la escuela con un ramo de flores para la maestra.

Todos los días, al salir de la oficina, iba su hijo Jesús a tomar el mate con ella.

En verano, después de cenar se sentaba en la vereda y hacía dormir en su falda a los dos hijos menores: Ema y papá.

\section{Otra hoja con notas sobre Isabel}

Abuela Isabel había mandado limpiar el aljibe de la casa. Cuando el obrero bajaba, se rompió la escalera y cayó al fondo. Abuela oyó sus gritos y, sin pedir ayuda, le tiró el balde para que se sostuviera y ella sola lo subió.

\bigbreak{}

Sus nietos recuerdan el dulce de boniatos que hacía: los cortaba en rodajas finas, y junto con su sabor lucía su aspecto: amarillo en almíbar transparente.

\bigbreak{}

Más ejemplos de su callada generosidad:

Avisan a la familia, radicada desde hacía ya varios años en Treinta y Tres, que han encontrado muerto en los montes de Olimar al sobrino Justo Núñez. Isabel compra su ataúd, y parte a acompañar a los dolientes, llevando lo necesario en ese momento.

\bigbreak{}

Su media hermana Dominga se había radicado en José Pedro Varela. Vivía allí modestamente. Isabel la visitaba con alguna frecuencia, trasladándose en motocar. Se preparaba para el viaje: gallinas, matambre y variedad de alimentos. Nunca decía que eran para la hermana\ldots y todos simulaban creer que en tan corto viaje consumía lo preparado.

\bigbreak{}

Isabel debía comer con muy poca sal, lo que le producía disgusto. Pedía a la empleada que le preparara un churrasquito, y se escondía en un maizal a saborearlo.

\bigbreak{}

Cuando contrajo la viruela era muy joven, y llevaba poco tiempo de casada. Perdió el cabello y las pestañas, y su rostro se veía hinchado y lleno de marcas. Cuando se vio\ldots lloraba sin consuelo.

En las fotos que hay de ella, se la ve como una anciana fea. Pero sus nietos nunca lo notaron. Para ellos, Mamá Isabel tenía toda la belleza del amor.

\subsection{Isabel ante la muerte de Jesús}

Cuando fue a visitarlo por última vez, aparentaba ante la familia ignorar la proximidad del fin, pero dijo a una empleada: ``Mi pobre hijo se termina, como una vela.''

Ya fallecido, no sollozaba ni manifestaba dolor desesperado: sentada en un sillón, rodeada de acompañantes, sufría en silencio aquella última y cruel herida de la vida.

\chapter{Difteria}

Para agregar a la historia de las 3 niñas muertas, junto con las partidas de defunción.

Cuando se produjo la epidemia de difteria, los hijos nacidos eran 7 u 8 (hay dudas sobre si Ernesto ya existía). A los mencionados en el capítulo correspondiente, debe agregarse Antonia, nacida probablemente en 1888 --- no figura en el archivo parroquial, al igual que sus hermanos inmediatos mayores, Isabel y Jesús.

Tres de ellos fallecieron en Olimar Chico, según lo narrado.

El dolor de los padres era tal, que Antonio, temeroso por la razón de Isabel, solicitó a una parienta: Rogelia Lámpez (¿tal vez hija suya?) que se hiciera cargo de la casa. Accedió y convivió con la familia hasta su muerte, ocurrida a avanzada edad.

Los Vázquez Eliseche se trasladan a treinta y Tres. Próximo ya el nacimiento de Siria (la 2$^{a}$ con ese nombre), se produce el fallecimiento de la cuarta hija. Por ser la causa enfermedad epidémica; no se permitió velatorio. Fueron a retirar el cuerpo, y el padre enajenado por el dolor, se opuso a entregarlo, defendiendo con un arma sus palabras; se manifestó dispuesto a velarla y sepultarla en la propia casa.

Fue menester la mediación de la familia para que depusiera su actitud.

\bigbreak{}

Perdieron otro hijo: un varón que falleció al nacer, y que iba a ser llamado Pedro, como su bisabuelo paterno.

\chapter{Siria --- Prima --- Ema Vázquez Eliseche}

De ellas, casi todo lo ignoro. De los tíos, solo Isabel las conoció.

Cuando los recuerdos llegaban a las hermanitas muertas, sus ojos se llenaban de lágrimas, lo que nos hacía desviar la conversación.

La muerte llegó primero a las dos menores. El temor al contagio hacía que, tan pronto como se producía un fallecimiento (¡y fueron 3 en 3 días!) Antonio, el padre, envolvía el cuerpo y lo sepultaba de inmediato, en el monte.

Ya radicada la familia en Treinta y Tres, los restos fueron trasladados\footnote{En 1895 figuran Siria y Prima}, y hoy reposan junto a sus padres y hermanos.

\bigbreak{}

Del Libro de Bautismos de la Parroquia de San José Obrero de Treinta y Tres:

\begin{itemize}
    \item{Marciala Ciria --- nacida el 29 de julio de 1880}
    \item{Feliciana Prima --- nacida el 9 de julio de 1882}
    \item{Eulogia Ema --- nacida el 29 de julio de 1890}
\end{itemize}

En la familia se las nombraba Ciria, Antonia y Ema. En los libros parroquiales no figura ninguna Antonia\footnote{Apareció como Juana, nacida el 29 de agosto de 1887} y sí Prima.

\chapter{Andrés Vázquez Eliseche}

22-VIII-1878 / 4-V-1899

\bigbreak{}

Veinte años. Vida breve que se apagó con el siglo.

Muy poco de él llegó hasta nosotros.

Que trabajaba en la Jefatura de Policía\ldots Que tocaba la guitarra\ldots Que era muy querido por todos\ldots

Escasos y pobres detalles, que no alcanzan para perfilar a ese joven que se asomó a la vida, y la dejó para entrar en la eternidad del tiempo.

Hoy no existe nadie de quienes le conocieron y le amaron. A sus sobrinos, no nos queda más que un nombre.

El hallazgo de un álbum mortuorio nos da un rostro, permite que nos asomemos a un dolor, nos fija una fecha.

Esto es todo, tío Andrés.

Hay una dedicatoria al dorso de la fotografía que dice: ``A mis queridos padres. Treinta y Tres, Junio 17 de 1898''.

\chapter{Jesús Vázquez Eliseche}

Me contó su hija Isabel: Jesús era severo y justo.\ 

Una persona que mimaba a sus hijos, pero tenía reacciones que mostraban su sentir; en una oportunidad Rosa llevó al liceo abrojos, y organizó una guerrilla en clase. Esto avergonzó a su padre; que la castigó con severidad. Al año siguiente, un profesor amigo de Jesús, tenía entre sus alumnos a Isabel. Este señor acostumbraba a contar chistes, y reía con el grupo. Pero cuando él dejaba de hacerlo, bruscamente, exigía que todos quedaran en absoluto silencio. Isabel continuó riendo\ldots y fue expulsada de clase. No se atrevió a contarlo, pero Jesús se enteró. Oyó sus explicaciones, y de inmediato escribió al profesor amigo una carta de rompimiento: nunca más lo trató.

\bigbreak{}

En una oportunidad pudo ver que un muchacho grande molestaba a su hijo Julio, que falto de oído, y sin poder hablar, no podía defenderse. Enojado lo corrió hasta que el imprudente se refugió en una casa, donde lo ocultaron.

Igual reacción tuvo cuando un compañero de Roberto lo lastimó en la escuela.

\bigbreak{}

\emph{(nota del transcriptor: acá probablemente falta algo)}

\bigbreak{}

Luego se dio cuenta del error, pero no lo informó: calladamente asumió la responsabilidad de pagarlo, hasta su muerte. Recién entonces el dueño del cine conoció el gesto de su amigo.

Con alguna frecuencia iba a pasar algunos días un hombre de apellido Mí\-guez, a quien ponían cama en el cuarto de los varones. Una vez los alarmaba di\-cién\-do\-les que estaba embrujado, que se sentía mal. Fastidiado, Jesús le dijo que le iba a dar un remedio muy efectivo contra las brujerías, y permaneció toda la noche levantado haciéndole tomar cucharadas de aceite crudo.

\bigbreak{} 

Jesús era liberal, pero respetaba las creencias de los demás. Uno de sus más queridos amigos, fue el sacerdote Ricardo Álvarez.

\bigbreak{}

Había sacado una lotería: \$30.000; una cantidad importantísima en aquella época. Le pidieron dinero prestado cantidad de personas, y él a ninguno se los negó.

Cuando poco tiempo después se casó su hija Isabel, el tío Ernesto pagó la fiesta: Jesús, nada tenía. Y para ir a Montevideo a cuidar su quebrantada salud, debió solicitar un préstamo al Banco. Nada reclamó a los deudores.

Al notar próximo su fin, quemó los papeles en los que había anotado en su momento los préstamos realizados.

Después de fallecido, sólo uno: José Helal, se presentó a Rosa, la viuda, comunicándole la deuda, y la imposibilidad de pagarla. Conociendo su difícil situación, Rosa lo liberó del compromiso.


\chapter{Luis Antonio Vázquez Eliseche --- Pepe}

Fue el menor de los hijos. Nació el 21 de junio de 1898. Tenía sólo 6 años cuando falleció su padre. Esto, y la marcada diferencia de edad con los hermanos mayores, le hizo el consentido de la familia: Se parecía al padre: muy alto, rubio, de ojos verdes y nariz aguileña. Fue muy bien parecido.

Su niñez y juventud fue distinta a la de sus hermanos: lo que en estos era seria responsabilidad, en él era alegre diversión; la postura severa frente a la vida de unos, era en él imprevisión y jarana. Amistad desprejuiciada, amoríos fáciles: eso fue su vida hasta los 36 años.

El 18 de agosto de 1934 se casa con Irma Landa Aguerre, y se produce el cambio. Radical. Excelente esposo, asumió todas las responsabilidades que había eludido. Se hizo hogareño, constante en el afecto y en el trabajo. Fue jefe de la Oficina de Aguas Corrientes ---después OSE---.\@ Ocupó desde su matrimonio la casa que el Organismo destinaba al jefe, en Manuel Meléndez N$^o$~440. Allí nacimos sus hijos, y allí falleció.

En 1942 fue trasladado a Durazno, donde permaneció 2 años retornando luego a Treinta y Tres.

De genio vivo y rápido, sus enojos eran breves. ¡Y con nosotros tenía la paciencia del mundo! Le gustaba inventar cuentos que nunca faltaban a la hora de dormir, en los que desfilaban animales y se sucedían aventuras.

El gusto por la pesca lo acompañó hasta el fin. Tenía su ``barra'' con la que periódicamente se iba de campamento. También le atraía la caza, pero en menor grado.

En 1957 fue operado de cáncer en las glándulas salivares. En mayo de 1962, enferma, aparentemente de hepatitis. Pero el 30 de julio de 1962, fallece de cáncer de hígado. Días antes había encontrado a Dios, y mientras puede hablar, repetía: ``¡Qué bueno es Dios, que me perdonó!!!''

\section{Anécdotas}

Nos hacía reír contando que cuando aprendió a manejar, tiró a un ``turco'' que llevaba su mercadería en un cajón, desparramando todo.

No le gustaba negarnos nada; cuando le pedíamos algo inconveniente, lo resolvía con un ---``Andá a pedírselo a Irma.''

Cuando Chacha o la abuela Isabel lo despertaban temprano para darle un medicamento indicaba: ¡dénselo a Ernesto!

Su letra era hermosa\footnote{En el original hay un texto escrito por él que dice ``Manuel Oribe entre M. Lavalleja y M. Meléndez. Treinta y Tres, Setiembre 30 de 1939'' y firma Luis A. Vázquez}.

\chapter{Marcelina Vázquez Eliseche --- Chela}

2-VI-1886 / 26-XII-1978

\bigbreak{}

Muy fuerte personalidad. El casamiento de Jesús, su hermano mayor, la colocó como centro de la familia, a la que manejó con mano dura.

Su inteligencia clara, y su visión realista de la vida, le permitieron suplir con éxito carencias de instrucción (cursó solo hasta 2$^o$ año de escuela rural). Durante su larga vida se mantuvo informada, emitiendo juicios precisos y terminantes sobre diferentes temas. Lectora ávida, sabía escuchar, tenía buena memoria y su conversación cautivaba. Su extraordinaria habilidad manual se manifestó en campos dispares: bordados, encajes, pintura, esterillado. Amaba las flores, que bajo su cuidado crecían\ldots sin orden mi concierto en el jardín más maravilloso que recuerdo.

Los años suavizaron su genio, pero no doblegaron su voluntad. Intentó ---infructuosamente--- oponerse al casamiento de sus hermanas Ema y Siria, pero luego fue cuñada afectuosa, conquistando cariño y respeto. Sus debilidades ---creo que las únicas--- fuimos sus sobrinos.

Los diez encontramos en ella la veta oculta de suavidad. Para nosotros eran las anécdotas de la niñez lejana, llenas de interés, que nos sumergían en épocas de revolución partidaria, con personajes que hoy son historia.

Aferrada en extremo a la casa paterna, recibía amistades y familiares, abandonándola solo para visitar enfermos o acompañar a muertos. Eran una unidad indisoluble y no se la concebía en otro ambiente, y la casa, sin ella, perdía el alma.

Vivió hasta los 92 años. Siempre activa; nunca cedió autoridad.

Pero dio tanto, tanto amor, que solo eso queda en el recuerdo. Habiendo sufrido fractura de cadera su firme voluntad la llevó a sobreponerse; logró volver a caminar, colocándose nuevamente a la cabeza de los hermanos.

La edad, el accidente sufrido, desgastaron su organismo, y el 26 de diciembre de 1978, su corazón cansado doblegó a su cuerpo, que se negó hasta el fin a aceptar la inevitable.

Chacha. Su muerte dividió nuestras vidas en dos etapas: cuando Chacha vivió\ldots después que Chacha murió.

\section{Anécdotas}

\begin{itemize}
    \item{De joven le gustaba divertirse. Una vez, en carnaval, transformó la casa en taller de costura; todas sus amigas se reunieron allí para hacer los trajes de margaritas que lucieron en un carruaje.}
    \item{Tenía 90 años, y hacía ravioles caseros, armándolos uno a uno, como pastelitos, para ponerles más relleno.}
    \item{Sabiendo que no me gustaba el café con leche (merienda obligatoria e infaltable en su casa), con frecuencia me llamaba a la cocina donde me hacía un huevo frito. Era un secreto entre las dos. Eso sí: después tenía que tomar la taza de café con leche.}
    \item{Con las podas de peral, hacía agujas de gancho con total perfección. Llamaban la atención las flores de tela que hacía para obsequiar a familiares y amistades.}
    \item{De la quinta y del jardín se ocupaba personalmente. No era raro verla, en sus últimos años, cavando y dando vuelta tierra.}
    \item{En época de higos, nos esperaba con una fuente honda colmada; la rodeábamos, y ella pelaba y nos lo ponía directamente, diciendo: ¡boca! (nosotros, obedientes y encantados, la abríamos lo más grande que podíamos).}
    \item{Todos los sábados íbamos a su casa\footnote{Aún ahora, al hablar de la casa que ya no nos pertenece, decimos ``la casa de Chacha''.}. Al principio, de mañana y pasábamos el día. Después, cuando ya los años se hacían sentir, de tarde. Temprano. Aún con los más grandes calores, o la lluvia más intensa. Y si nos retrasábamos, desde lejos la veíamos en la esquina, inquieta y preocupada.}
    \item{Si pasábamos el día con ella, ¡siesta obligatoria! Nunca dejó de acostarse después del almuerzo, hasta las tres y media o cuatro de la tarde.}
    \item{En una oportunidad me reconstruyó una muñeca de porcelana. Le hizo peluca con rulos, de pelo natural, la pintó, y le cosió un hermoso vestido celeste, con encajes. ¡Más linda que cuando era nueva!}
    \item{Con mucha frecuencia hacía tortas fritas por la tarde. Y si había niños, modelaba con masa palomitas que eran una delicia de ver\ldots y de comer.}
    \item{Aunque eran muy ancianos, allí cenaban todos. ¡Y cómo! ¡Hasta porotos con carne de cerdo!}
\end{itemize}

\chapter{Lázara Siria Vázquez Eliseche}

17-XII-1893 / 2-II-1979

\bigbreak{}

\emph{Una nota escrita a mano dice ``También estudió piano, ya maestra''}

\bigbreak{}

Fue; al igual que Ema, maestra, carrera que ejerció con gran vocación. Sé que trabajó en la escuela N$^o$~1, y también en la N$^o$~31, de la que era directora en el momento de jubilarse. También fue profesora de Idioma Español en el liceo de Treinta y Tres. De inteligencia normal, tuvo gran voluntad y responsabilidad en el trabajo, lo que la llevó a destacar en sus actividades.

A diferencia de su hermana Isabel, tenía carácter suave. Muy cariñosa con todos; su bondad para con su madre fue algo especial.

También sus sobrinos recibimos su ternura. Los de nuestra rama le de\-cí\-a\-mos Chiá. Era morena, de cabello lacio y fino, grandes ojos oscuros y estatura mediana.

Después de jubilada vivió unos años en Montevideo. Se casó por esa época con Doroteo Furtado, viudo con una hija. El matrimonio pronto fracasó. El divorcio, no muy frecuente en ese tiempo, la hirió profundamente.

Ante problemas de salud de sus hermanos, retornó a Treinta y Tres, pero añoró siempre la vida de la capital. Nuevamente la fuerte personalidad de Chacha absorbió la suya, transformándose en una sombra resignada, volcada a los sobrinos, y con esporádicos arrebatos de mal genio, en los que se traslucía toda la rebeldía de la frustración.

Se fracturó la cadera antes que Chacha, pero no tuvo la fuerza de voluntad de ella. Logró trasladarse en andador durante breve lapso, pero luego se dejó vencer. Falleció el 2 de Febrero de 1979, poco más de un mes después que Chacha. El consuelo: había encontrado a Dios, poco antes de morir.

\emph{En el original está escrito con la letra de Siria lo siguiente: ``Un beso fuerte de esta Chiá que tanto los quiere'' y firmado ``Siria''.}

\section{Anécdotas}

\begin{itemize}
    \item{Era muy aprensiva. Cuando comenzó a estudiar debió trasladarse a Montevideo. Por ese tiempo el viaje se hacía ---no sé si en su totalidad, o sólo hasta Nico Pérez--- en diligencia. La acompañó la abuela Isabel. Frente a ellas se sentaba un hombre cuyo aspecto la atemorizó. ¡Lloró todo el trayecto porque quería bajarse!}
    \item{Después, ya en Montevideo, también lloró hasta que obtuvo el título, porque extrañaba a su madre.}
    \item{Estaba durmiendo, y se le durmió una mano. La tomó con la otra y al  no sentirla despertó gritando: ---¡Chela! ¡Enciende la luz que agarré un ladrón!}
    \item{Fue una de las primeras mujeres que tuvo auto en Treinta y Tres, y la primera en conducirlo.}
\end{itemize}

\chapter{Plácido Ernesto Vázquez Eliseche}

5-X-1892 / 28-IX-1980

\bigbreak{}

¡Qué nombre tan bien puesto! Plácido: así fue.

Sólo hay de él recuerdos gratos. En nuestra niñez, juguetón y alegre. Luego, el tío con el que pasaban sin sentir las horas, pues de sus labios brotaban siempre las anécdotas, los cuentos oportunos, que últimamente se repetían, pero siempre entretenían.

Su vida no fue lo feliz que merecía su bondad.

Inteligencia preclara, espíritu sensible, amante fiel de toda manifestación cultural. Cuando cursaba 2$^o$ año liceal, su salud quebrantada requería cuidados especiales. El fundado temor de la madre, tan golpeada por repetidas desgracias, hizo que abandonara el estudio.

Esto debió ser un gran dolor par aquel adolescente serio y ávido de saber.

Rodeado de cuidados, centro de cariño, protección y zozobras, transcurrió su larga vida. Extremadamente delgado, explotó sanamente la atención familiar de la que era objeto, cultivando ingenuos caprichos: pobre compensación por una existencia dependiente.

A lo que no renunció, fue al estudio ya no sistematizado, pero sí bien orientado y constante. Podía mantener largas conversaciones sobre historia, ilustradas con anécdotas de los personajes; actualizaba sus conocimientos geográficos, señalando meticulosamente en el atlas, por ejemplo, las colonias africanas que se independizaban, reteniendo nombres, capitales, ubicación, cambios de gobierno, etc. Pocos meses antes de fallecer, deseó adquirir un diccionario castellano actualizado, porque ``La Real Academia ha incorporado muchos términos de los que no estoy al tanto''. Memoria privilegiada, repetía poemas de clásicos españoles y americanos. Sus autores preferidos: Víctor Hugo, Almafuerte, José Santos Chocano, José Hernández. Tampoco le fue extraña la medicina, cuyos adelantos seguía con interés.

Conoció el amor. Una maestra, de apellido Correa. Pero también aquí fue derrotado. Por un lado la sumisión, que ya era carne en él, lo llevó a acatar la oposición de Chacha, siempre temerosa por su salud y bienestar. Por otro lado, la conducta poco digna de la novia. Y volvió al mundo pequeño del hogar y de los libros.

En su juventud ejerció un cargo administrativo en la jefatura de Policía, en el que se jubiló prematuramente.
Como distracción, y para aumentar sus modestos ingresos, instaló una academia de dactilografía y taquigrafía.
Su afán perfeccionista le llevó a crear un método en esta última técnica, el que le valió ser nombrado miembro honorario de la Sociedad Mundial de Taquígrafos, con la que se comunicaba regularmente.
Con modestia, evitaba hablar de este tema.

A edad avanzada fue nombrado Sub-Jefe de policía, cargo en el que cesó cuando se requirió grado policial para ejercerlo.

Apasionado político, militó ---como toda su familia---, en el Partido Colorado. Pero era demasiado puro su idealismo, y cedió ante la realidad de los intereses, retirándose también de esta actividad, si bien se mantuvo hasta el fin, al tanto de los menores detalles.

Cuando quedó solo, también quebrado de la cadera, buscó un nuevo interés: el cultivo de frutales. Esto, sus anhelos inquebrantables de vivir, y el cariño de todos los que le trataron, hicieron que ese anciano tan delgado, sobreviviera a todos sus hermanos. También él había hallado a Dios y en Él descansó el 28 de setiembre de 1980.

Con él se fue el último. Pasaron a ser sombras. Hoy todavía recordados, respetados, pero que se sumergen cada vez más en el ayer, hasta que llegará un día en que solo estarán en nuestros corazones, llenando de nostalgia los recuerdos de una época feliz en que su cariño era respaldo, y sus vidas, ejemplos.

\section{Anécdotas}

Su falta de apetito era el martirio de la Chacha; le hacía solo lo que más le gustaba. Aún así, a la hora de comer, tomaba su sombrero (jamás salió sin él) y desaparecía. Cuando regresaba, y no tenía más remedio que sentarse a la mesa, comenzaba a revolver en el plato, apartando lo que no iba a comer. ¡Y era casi todo!

Le gustaban las bananas. Prepararlas era toda una ceremonia: con una cortaplumas ---nunca un cuchillo--- cortaba los extremos; luego hacía un corte longitudinal, y retiraba la cáscara. Más tarde la raspaba para quitar las fibras. Si algo quedaba, era lo que cortaba en rodajitas e iba comiendo, con total serenidad, y masticando un número infinito de veces.

El sagrado café con leche\ldots rara vez lo tocaba. ``Hoy quería té'' o mate cocido, o leche con copos de maíz. Como demoraba en tomarlo, se enfriaba, pedía que la calentaran, y se iba. Cuando regresaba, ¡otra vez frío!

Para comer pan, o queso, los cortaba (siempre con la cortaplumas) en dados pequeños que consumía uno a uno.

Cuando ya estaba solo, y poco podía caminar, tomaba taxi hasta para ir al centro a comprar una birome. A veces aparecía en casa con la primera sandía, o una botella de vino espumante, o con cualquier regalo que se le ocurría.

Encargaba árboles frutales a Estados Unidos, y luego se entretenía cui\-dán\-do\-los con esmero\ldots pero no los vio crecidos. Pese a que decía ---``Y pienso verlos dar fruto'', Dios dispuso otra cosa.

Tenía siempre almanaques con espacios para anotaciones diarias.

\chapter{Modificaciones}

Las investigaciones en los libros de bautismos de la Parroquia San José Obrero, de Treinta y Tres, introdujeron cambios en el orden cronológico de los hijos de Antonio Vázquez e Isabel Eliseche, conocido por mí por conversaciones de familia.

No hay constancias de los bautismos de Jesús e Isabel, pero obtuve las fechas en la Intendencia.

Resulta entonces el siguiente ordenamiento por edades:

\begin{table}[htb]
    \centering
    \begin{tabular}{l r}
        Nombres & Fecha de nacimiento\\
        \midrule
        Andrés & 22-VII-1878\\
        Marciala Ciria & 29-VI-1880\\
        Feliciana Prima & 9-VI-1882\\
        Julio Jesús & 20-VI-1884\\
        Marcelina Isabel & 2-VI-1886\\
        \multicolumn{2}{c}{Aquí deben estar Antonia y Pedro}\\
        Eulogia Ema & 11-III-1890\\
        Plácido Ernesto & 5-X-1892\\
        Lázara Siria & 17-XII-1893\\
        Benita Ema & 12-I-1896\\
        Luis Antonio & 21-VI-1898\\
    \end{tabular}
\end{table}

Antonia aparece en el Reg. Civil como Juana, nacida el 29 de agosto de 1887. De Pedro, no aparece nada.

\chapter{Poemas}

\section{Añoranza}

\begin{verse}[\textwidth]

Casona triste y muda, en ti se duerme el tiempo.\\
En tus paredes viejas conservas las historias\\
que ha poco repetían los labios que se fueron.\\
En tus rincones yacen perdidos, los recuerdos\\
de los felices días en que aún palpitaban,\\
henchidos de cariño, corazones ya viejos.\\
Te habitan los espectros de tíos y de abuelos.\\
que vagan por tus piezas, soñando los recuerdos\\
de horas que no vuelven, de amores que murieron.

\end{verse}

\emph{(Incluye una foto de la casa. No dice quién lo escribió.)}

\newpage{}

\section{La niña muerta}
\begin{verse}[\textwidth]

Delicada muñequita,\\
¿quién te vistió tan bonita\\
para este mundo dejar?

¡Qué pena me dan tus ojos\\
que a ninguna parte miran\\
porque no pueden mirar!

¡Qué desamparo en tu rostro\\
y en tus manitas crispadas\\
por aferrarse a la vida\\
que termina de escapar!

¿Quién en su pena infinita,\\
muerta, te quiso guardar\\
en esta imagen dolida\\
que simboliza la angustia\\
de una muerte colectiva?

Y es una muda protesta\\
la brevedad de tu vida.

\end{verse}
\attrib{Isabel Vázquez Brovia}

\end{document}
